% !TeX spellcheck = en_US 
\documentclass[report]{byu-aero}

%This sets the visible levels shown on the Table of Contents. 2 will show the Sections and Subsections. This seems okay to save some space. If you want to show subsubsections change it to 3. 1 will just show the sections, but that seems too brief.
\setcounter{tocdepth}{2}

%!!!!!!!!!!!!!!!!!!!!!!!!!!!!!!!!!!!!!!!!!!!!!!!!!!!!!!!!!!!!!!!!!!!!!!!!!!!!!!!!!!!!!!!!%
%!!!!!!! UNLESS YOU KNOW WHAT YOU'RE DOING, DO NOT TOUCH ANYTHING ABOVE THIS LINE !!!!!!!%
%!!!!!!!!!!!!!!!!!!!!!!!!!!!!!!!!!!!!!!!!!!!!!!!!!!!!!!!!!!!!!!!!!!!!!!!!!!!!!!!!!!!!!!!!%

%%%%%%%%%%%%%%%%%%%%%%%%%%%%%%%%%%
%%%%%%%%%%   Text Body   %%%%%%%%%
%%%%%%%%%%%%%%%%%%%%%%%%%%%%%%%%%%
\begin{document}

% Title Page
\includepdf[page=-]{titlepage.pdf}

%%%%%%%%%%%%%%%%%%%%%%%%%%%%%%%%%%%%%%%%%%
%%%%%%%%%%   Table of Contents   %%%%%%%%%
%%%%%%%%%%%%%%%%%%%%%%%%%%%%%%%%%%%%%%%%%%
\setcounter{page}{2} %This sets the page counter to 2, since the Title page is counted as page 1
\thispagestyle{tocpage} %This uses the alternate table of contents page header style
\tableofcontents %This makes the table of contents
%The following 3 lines add the Table of Contents "section" to the Table of Contents
\phantomsection
\addcontentsline{toc}{section}{Table of Contents}
\markboth{Table of Contents}{Table of Contents}

\clearpage
\newpage

%%%%%%%%%%%%%%%%%%%%%%%%%%%%%%%%%%%%%%%%%%
%%%%%%%%%%   Executive Summary   %%%%%%%%%
%%%%%%%%%%%%%%%%%%%%%%%%%%%%%%%%%%%%%%%%%%
\section{Executive Summary (5 Points)}
\label{sec:ExecutiveSummary}
\begin{itemize}
\item Maximum of 1 page. If exceeded, score as 0 points
\item Summary description of selected design and why it best meets the mission requirements
\item Main points from subsequent sections
\item Document the performance/capabilities of your system solution
\end{itemize}

% Performance Summary Table
\begin{table}[h!]
	\centering
	\caption{Summary of major system perfomance factors.}
	\label{tab:performancesummary}
	\rowcolors{2}{BYUbluelite}{white}
	\begin{tabular}{ |c|c| } 
		\hline
		\rowcolor{BYUbluemid}
    	Metric & \\ 
		\hline
	     & Performance (units) \\ 
		\hline
		 & Performance (units) \\ 
		\hline
	\end{tabular}
\end{table}

%%%%%%%%%%%%%%%%%%%%%%%%%%%%%%%%%%%%%%%%%%%
%%%%%%%%%%   Management Summary   %%%%%%%%%
%%%%%%%%%%%%%%%%%%%%%%%%%%%%%%%%%%%%%%%%%%%
\section{Management Summary (5 Points)}
\label{sec:ManagementSummary}
%\begin{itemize}
%\item Describe the organization of the design team
%\item Chart of design personnel and assignments areas
%\item Milestone chart showing planned and actual timing of major elements
%\end{itemize}

Paragraph describing the organization of the design team, citing \cref{fig:personnelassignments}.

\begin{figure}[h!]
	\centering
	\includegraphics[width=3.5in]{draft.pdf}
	\caption{This chart depicts the design personnel and assignment areas within our team structure.}
	\label{fig:personnelassignments}
\end{figure}

2nd paragraph about milestone chart shown in  \cref{fig:plannedvsactualtiming} (be brief).

\begin{figure}[h!]
	\centering
	\includegraphics[width=3.5in]{draft.pdf}
	\caption{This milestone chart reveals our original plan for major elements of our design process compared to the actual timing of these events.}
	\label{fig:plannedvsactualtiming}
\end{figure}

%%%%%%%%%%%%%%%%%%%%%%%%%%%%%%%%%%%%%%%%%%
%%%%%%%%%%   Conceptual Design   %%%%%%%%%
%%%%%%%%%%%%%%%%%%%%%%%%%%%%%%%%%%%%%%%%%%
\section{Conceptual Design (15 Points)} %Note: this should be completed already as part of the proposal
\label{sec:ConceptualDesign}
% \begin{itemize}
% \item Describes mission requirements (problem statement)
% \item Translate mission requirements into sub system design requirements
% \item Present a scoring sensitivity analysis.
% \item Review solution concepts/configurations considered
% \item Describe concept weighting and selection process and results
% \end{itemize}

\subsection{Mission Requirements}
\label{ssec:missionreqs}

Describes mission requirements (problem statement)

Translate mission requirements into sub system design requirements
\subsubsection{Aerodynamic Requirements}
\label{sssec:aeroreqs}

\subsubsection{Structural Requirements}
\label{sssec:structreqs}

\subsubsection{Propulsion Requirements}
\label{sssec:propreqs}

\subsubsection{Specialty Requirements} %change this to be specifically what it is (like bomb drop or whatever the specific mission is)
\label{sssec:specialreqs}

\subsection{Scoring Sensitivity Analysis}
\label{ssec:scoringsensitivity}

Present a scoring sensitivity analysis.

\subsection{Concept Weighting and Selection Process}
\label{ssec:selectionprocess}

Review solution concepts/configurations considered

Describe concept weighting and selection process and results
 
% Figure of Merit (i.e. the weights used in decision matrices)
\begin{table}[h!]
	\centering
	\caption{Figures of Merit}
	\label{tab:fom}
	\rowcolors{2}{BYUbluelite}{white}
	\begin{tabular}{ |c|c| } 
		\hline
		\rowcolor{BYUbluemid}
    	Factor & Relative Importance (1-5) \\ 
		\hline
	     &  \\ 
		\hline
		 &  \\ 
		\hline
		 &  \\ 
		\hline
		 &  \\ 
		\hline
		 &  \\ 
		\hline
	\end{tabular}
\end{table}

%Example Decision Matrix.  Probably going to have a few of these.
\begin{table}[h!]
	\centering
	\caption{Weighted decision (Pugh) matrix.}
	\label{tab:decisionmatrix1}
	\rowcolors{2}{BYUbluelite}{white}
	\begin{tabular}{ |c|c|c|c|c| } 
		\hline
		\rowcolor{BYUbluemid}
    	Factor & Weight & Option 1 & Option 2 & Option 3 \\ 
		\hline
	     &  &  &  &  \\ 
		\hline
		 &  &  &  &  \\ 
		\hline
		 &  &  &  &  \\ 
		\hline
		 &  &  &  &  \\ 
		\hline
		\multicolumn{2}{|c|}{Totals} &  &  &  \\ %BOLD WINNING OPTION
		\hline
	\end{tabular}
\end{table}

%Figure of rejected concepts (this may be better as several figures showing options for different aspects of the design.
\begin{figure}[h!]
	\centering
	\includegraphics[width=3.5in]{draft.pdf}
	\caption{Here we show a sampling of the design concepts we rejected along the way as we honed in on our final design concept (see \cref{fig:finalconcept}).}
	\label{fig:rejecteddesigns}
\end{figure}

\subsubsection{Final Concept}
\label{sssec:finalconcept}

%Figure of the final concept
\begin{figure}[h!]
	\centering
	\includegraphics[width=3.5in]{draft.pdf}
	\caption{Our final conceptual design incorporates the highest scoring options in the decision matrices described above.}
	\label{fig:finalconcept}
\end{figure}

%%%%%%%%%%%%%%%%%%%%%%%%%%%%%%%%%%%%%%%%%%%
%%%%%%%%%%   Preliminary Design   %%%%%%%%%
%%%%%%%%%%%%%%%%%%%%%%%%%%%%%%%%%%%%%%%%%%%
\section{Preliminary Design (20 Points)}
\label{sec:PreliminaryDesign}
% \begin{itemize}
% \item Describe design/analysis methodology
% \item Document design/sizing trades
% \item Describe/document methodology for prediction of aircraft performance (include capabilities and uncertainties)
% \item Provide estimates of the aircraft lift, drag and stability characteristics and method of prediction
% \item Provide estimates of the aircraft mission performance 
% \end{itemize}

\subsection{Methodology}
\label{ssec:methodology}

Describe design/analysis methodology

\subsection{Trade Studies}
\label{ssec:tradestudies}

Document design/sizing trades

\subsection{Estimated Aircraft Performance}
\label{ssec:estaircraftperfomance}

Describe/document methodology for prediction of aircraft performance (include capabilities and uncertainties)

\subsubsection{Uncertainty Analysis}
\label{sssec:uncertaintyanalysis}

Describe the capabilities and uncertainties of the tools used for performance estimation.

\subsubsection{Lift and Drag}
\label{sssec:liftdrag}

Provide estimates of the aircraft lift, drag and stability characteristics and method of prediction

\subsubsection{Stability}
\label{sssec:stability}

\subsubsection{Mission Performance}
\label{sssec:missionperformance}

Provide estimates of the aircraft mission performance 

%%%%%%%%%%%%%%%%%%%%%%%%%%%%%%%%%%%%%%%%%%%
%%%%%%%%%%%%   Detail Design   %%%%%%%%%%%%
%%%%%%%%%%%%%%%%%%%%%%%%%%%%%%%%%%%%%%%%%%%
\section{Detail Design (15 Points + 15 Points for Drawing Package)}
\label{sec:detaildesign}
% \begin{itemize}
% \item Document dimensional parameters of final design
% \item Document structural characteristics/capabilities of final design
% \item Document systems and sub-systems selection/integration/architecture
% \item Document Weight and Balance for final design
% \item Must include Weight \& Balance table empty and with each possible payload/configuration
% \item Document flight performance parameters for final design
% \item Document mission performance for final design
% \item Drawing package:
% 	\begin{itemize}
% 	\item 3-View drawing with dimensions of all configurations
% 	\item Structural arrangement drawing
% 	\item Systems layout/location drawing
% 	\item Payload(s) accommodation drawing(s)
% 	\end{itemize}
% \end{itemize}

\subsection{Sizing}
\label{ssec:sizing}

Document dimensional parameters of final design

\subsection{Structures}
\label{ssec:structures}

Document structural characteristics/capabilities of final design

\subsection{System Selection, Integration, and Architecture}
\label{ssec:systemdetails}

Document systems and sub-systems selection/integration/architecture

\subsection{Weights and Balance}
\label{ssec:weightsandbalance}

Document Weight and Balance for final design

Must include Weight \& Balance table empty and with each possible payload/configuration

%Weight and Balance Table
\begin{table}[h!]
	\centering
	\caption{Weight and Balance table including empty aircraft and each possible configuration.}
	\label{tab:wieghtsandbalance}
	\rowcolors{2}{BYUbluelite}{white}
	\begin{tabular}{ |c|c|c| } 
		\hline
		\rowcolor{BYUbluemid}
    	Configuration & Weight (grams) & CG Location (mm) \\ 
		\hline
	    Empty &  &  \\ 
		\hline
		Config 1 &  &  \\ 
		\hline
		Config 2 &  &  \\ 
		\hline
	\end{tabular}
\end{table}

\subsection{Flight Performance Parameters}
\label{ssec:flightperformanceparams}

Document flight performance parameters for final design

\subsection{Mission Performance}
\label{ssec:missionperformance}

Document mission performance for final design
 
\subsection{Drawing Package}
\label{ssec:drawings}

The following are drawings including a 3-View drawing with dimensions of all configurations, a structural arrangement drawing, a systems layout/location drawing, and payload accommodation drawings.

% 3-view drawing
\includepdf[page=-]{draft.pdf}
% Structural Arrangement Drawing
\includepdf[page=-]{draft.pdf}
% Systems Layout/Location Drawing
\includepdf[page=-]{draft.pdf}
% Payload Accommodation drawing
\includepdf[page=-]{draft.pdf}

%%%%%%%%%%%%%%%%%%%%%%%%%%%%%%%%%%%%%%%%%%%
%%%%%%%%%%   Manufacturing Plan   %%%%%%%%%
%%%%%%%%%%%%%%%%%%%%%%%%%%%%%%%%%%%%%%%%%%%
\section{Manufacturing Plan (5 Points)}
\label{sec:ManufacturingPlan}
% \begin{itemize}
% \item Document the process selected for major component manufacture
% \item Manufacturing processes investigated and selection process and results
% \item Manufacturing milestones chart: plan and actual
% \end{itemize}

Document the process selected for major component manufacture

Manufacturing processes investigated and selection process and results

Manufacturing milestones chart: plan and actual

% Figure of Merit (i.e. the weights used in decision matrices)
\begin{table}[h!]
	\centering
	\caption{Figures of Merit}
	\label{tab:fomman}
	\rowcolors{2}{BYUbluelite}{white}
	\begin{tabular}{ |c|c| } 
		\hline
		\rowcolor{BYUbluemid}
    	Factor & Relative Importance (1-5) \\ 
		\hline
	     &  \\ 
		\hline
		 &  \\ 
		\hline
		 &  \\ 
		\hline
		 &  \\ 
		\hline
		 &  \\ 
		\hline
	\end{tabular}
\end{table}

%Example Decision Matrix.  Probably going to have a few of these.
\begin{table}[h!]
	\centering
	\caption{Weighted decision (Pugh) matrix for manufacturing plan.}
	\label{tab:decisionmatrixmanufacturing}
	\rowcolors{2}{BYUbluelite}{white}
	\begin{tabular}{ |c|c|c|c|c| } 
		\hline
		\rowcolor{BYUbluemid}
    	Factor & Weight & Option 1 & Option 2 & Option 3 \\ 
		\hline
	     &  &  &  &  \\ 
		\hline
		 &  &  &  &  \\ 
		\hline
		 &  &  &  &  \\ 
		\hline
		 &  &  &  &  \\ 
		\hline
		\multicolumn{2}{|c|}{Totals} &  &  &  \\ %BOLD WINNING OPTION
		\hline
	\end{tabular}
\end{table}

\begin{figure}[h!]
	\centering
	\includegraphics[width=3.5in]{draft.pdf}
	\caption{This milestone chart reveals our original plan for major elements of our manufacturing process compared to the actual timing of these events.}
	\label{fig:plannedvsactualtimingmanufacturing}
\end{figure}


%%%%%%%%%%%%%%%%%%%%%%%%%%%%%%%%%%%%%
%%%%%%%%%%   Testing Plan   %%%%%%%%%
%%%%%%%%%%%%%%%%%%%%%%%%%%%%%%%%%%%%%
\section{Testing Plan (5 points)}
\label{sec:TestingPlan}
%\begin{itemize}
%\item Describe all major ground and flight tests performed.
%\item Objectives and schedule for each.
%\item Data to be collected and how applied.
%\item Test and flight check lists
%\end{itemize}

\subsection{Completed Testing}
\label{ssec:completedtesting}

Describe all major ground and flight tests performed.

Objectives and schedule for each.

Data to be collected and how applied.

\subsubsection{Ground Testing}
\label{sssec:groundtesting}

\subsubsection{Flight Testing}
\label{sssec:flighttesting}

\subsection{Planned Testing}
\label{ssec:plannedtesting}

Objectives and schedule for each.

Data to be collected and how applied.

\subsection{Test and Flight Checklists}


%%%%%%%%%%%%%%%%%%%%%%%%%%%%%%%%%%%%%%%%%%%%
%%%%%%%%%%   Performance Results   %%%%%%%%%
%%%%%%%%%%%%%%%%%%%%%%%%%%%%%%%%%%%%%%%%%%%%
\section{Performance Results (10 Points)}
\label{sec:PerformanceResults}
\begin{itemize}
\item Describe the demonstrated performance of key subsystems following execution of testing plan
\item Compare to predictions and explain any differences and improvements made
\item Describe the demonstrated performance of your complete aircraft solution
\item Compare to predictions and explain any differences and improvements made
\end{itemize}


%%%%%%%%%%%%%%%%%%%%%%%%%%%%%%%%%%%%%
%%%%%%%%%%   Bibliography   %%%%%%%%%
%%%%%%%%%%%%%%%%%%%%%%%%%%%%%%%%%%%%%
%Bibliography (5 Points)
%\item List of all published works referenced in the text must be present in this section.
%\item Any material taken from a published source in all previous sections must have a numerical subscript corresponding to the appropriate citation in this section.
%\item References should appear in numerical order.
%\item Format should match AIAA provided guidelines:
%\newpage
%\clearpage
%\bibliography{ref}{}
%\bibliographystyle{aiaa}

\end{document}