% !TeX spellcheck = en_US 
\documentclass[report]{byu-aero}

\title{\color{BYUblue} A very brief guide to designing, building, and flying an RC airplane}
\author{Judd Mehr, }
%!!!!!!!!!!!!!!!!!!!!!!!!!!!!!!!!!!!!!!!!!!!!!!!!!!!!!!!!!!!!!!!!!!!!!!!!!!!!!!!!!!!!!!!!%
%!!!!!!! UNLESS YOU KNOW WHAT YOU'RE DOING, DO NOT TOUCH ANYTHING ABOVE THIS LINE !!!!!!!%
%!!!!!!!!!!!!!!!!!!!!!!!!!!!!!!!!!!!!!!!!!!!!!!!!!!!!!!!!!!!!!!!!!!!!!!!!!!!!!!!!!!!!!!!!%

%%%%%%%%%%%%%%%%%%%%%%%%%%%%%%%%%%
%%%%%%%%%%   Text Body   %%%%%%%%%
%%%%%%%%%%%%%%%%%%%%%%%%%%%%%%%%%%
\begin{document}
\maketitle
\section{Clearly define your goals}

There must be a reason you want to design an airplane. What is it? In other words, what do you want the airplane to be able to do after you've designed and built it? These are many times called mission requirements. You can have a wide variety of mission requirements such as take-off weight, take-off distance, speed, flight range, flight time, payload requirements, payload dropping, modularity, or almost anything else.  You must identify all of the mission requirements before you begin anything else. 

\subsection{Translate your Mission Requirements into sub-system requirements}

Once you know what you want your plane to do, you need to translate that into what the various subsystems need to be able to do. You'll probably want to make these somewhat parametric in nature. For example, rather than saying that your propulsion system needs to provide x units of thrust, you could put that in terms of aircraft weight and drag. You'll want to get as detailed as you can here so that you have a clear direction for all of your various sub-systems going forward. Remember, you haven't made any design decisions at this point, you've simply stated what your aircraft needs to be able to do, and what the sub-systems need to be able to do to achieve the aircraft goals.

\section{Conceptual Design}

After you've laid out all your goals and requirements, you're ready to start making some decisions. This first design stage is called conceptual design for a reason, that is, you are only worried about concepts. This stage tends to be more qualitative than quantitative. The key tool being \href{https://en.wikipedia.org/wiki/Decision_matrix}{decision matrices}, or Pugh matrices. This is a decision method that weights the importance of different factors and then compares options with each of these factors. It helps to show what choice is best based on what is most important to you. You can apply such matrices to any part of the aircraft and it's subsystems.  You can apply it to configuration, propulsion, structures, airfoil selection, etc. It is most applicable, however, to the qualitative decisions as it applies a virtual quantity in the form of your weighted importance. When it comes to quantitative decisions, you'll probably want to wait for the preliminary design stage.

Another aspect of the conceptual design phase are ``back of the envelope'' calculations. These are things like wing sizing, and rough tail volume calculations. For example, if you had a requirement for weight and stall speed, you could find out what the product of your wing lift coefficient and planform area need to be.  Perhaps you already found that as you were translating your mission requirements into sub-system requirements. Now it would be time to make a conceptual decision about your planform design. You'll probably have some sort of constraint on span, and you'll want to make the chord such that the Reynolds number isn't too small, and that your required lift coefficient isn't impossibly high. You may even consider using a decision matrix to explore the trade-off between wing size and other variables. Again, this is just one small example; there is much more to think about, and it will depend on your mission requirements. A free resource for some of these kinds of things is the \href{https://byu.app.box.com/v/me415book}{ME415 book}. Google is also a good resource, and the FLOW Lab grad students will likely have some other good references. You'll want to have all of these back of the envelope calculations done before you move on to preliminary design. In fact, you'll want to have sufficient information that you could build your first prototype.

\subsection{Conceptual Prototyping and Testing}

\subsubsection{Concept CAD}
As you make conceptual decisions, you will want to begin creating a virtual prototype in CAD. This will be a source of distilled information that you can use to make a physical prototype for initial glide, or even full flight, testing. The earlier you start on CAD, the easier the next iterations and final product will be.

\subsubsection{Physical Concept}
In similar manner, when you begin creating physical prototypes early, you can quickly see places where you may have gone wrong with your design, additions that need to happen, and/or changes that need to be made. You should begin your first prototype immediately after your concept design review.  This prototype should probably be a set of sub-system prototypes, not necessarily put together at this point.On the aerodynamics side of things, you'll probably want something that can at least glide, if not fly. Perhaps you have a specific payload requirement. A prototype for that might not be joined up with an airframe at this stage, but you'll still want to start prototyping. At this stage, you want to make sure that each sub-system isn't ``waiting'' for some other subsystem to figure things out. You can make simple prototypes now, and worry about the details at later design stages. Your goal is to do subsystem testing on your concept to know what needs to be changed, or focused on during your preliminary design. 

\section{Preliminary Design}

In the preliminary design, you start getting into design tools. These are low-fidelity tools such as XFOIL, XFLR5, AVL, OpenVSP, etc.  By now, you'll have a full concept, CAD and physical model with some intuition about what to focus on. It is in this phase that you begin to put numbers behind the majority of your design decisions. For example, what airfoil(s) are you choosing for the wing(s)? Why? By the end of this stage, you'll basically be close to where the ME415 class is at the end of their semester, but perhaps without a second prototype, which you'll want to start immediately after the preliminary design review.

\subsection{Preliminary Prototyping and Testing}
Again, you should be updating your CAD model as you go.  Don't forget to save versions of your model so you can show comparisons at the end of your project.  Also, you should make sure to take pictures and video of physical models in case they break, and make sure to take notes about what went well, what didn't, etc. To put it simply, keep a log book of your prototyping and testing efforts.

After your preliminary design review, you'll want to create another prototype. This one will be a lot more refined than the concept prototype, since you've done more advanced analyses on the airframe and other sub-systems.  You still might not have all of the sub-systems integrated, but you should be working toward that goal.

\section{Detailed Design}
In the detailed design phase, you move on to high-fidelity modeling. You'll want to do some CFD and FEA analyses, and validate them with physical testing. Note that you probably won't have time to do detailed design on every aspect of the aircraft. You should focus your time on the critical systems (those things that you identified as most important in the conceptual design phase based on the sub-system requirements you created before that). 

\subsection{Detailed Prototyping, Testing, and Validation}
In this prototyping phase, which begins right after your critical design review, you'll want to be completely integrating all of the sub-systems. On the other hand, you may need to do some testing on specific sub-systems in order to validate your detailed design. This might include wind tunnel testing of a payload component, all the way up to a scale model of the entire airframe. It may include some sort of structural testing, such as a 3-point load test or some tensile/compression tests on part of a composite layup.  Again, you won't be able to do everything with the time you have, so you should choose only the most important things, or those items that you cannot develop a virtual model to analyze. 

You may find that you create a second prototype in this phase since your on paper design will be complete, but you may (will likely) learn some things in flight testing that you didn't anticipate. A second prototype will also be good practice for the final manufacture of the whole system, and, in a pinch, could be used as the final product if you run out of time. (You should plan not to, but life happens...) You will want this prototype to be the complete product, and you'll want to fly it through the defined mission(s) at least a few times to make sure everything is behaving as expected.  

\section{Final Manufacture and Flight Test}
This is where you make your airplane completely, and as perfectly, and aesthetically as you can.  You then fly it, take a video as proof of flight, and probably run through the entire mission at least a few more times.

\section{Fly Off}
Go take your plane and show it off!


\end{document}